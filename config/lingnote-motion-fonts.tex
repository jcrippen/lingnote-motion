%!TEX root = ../lingnote-motion.tex
%!TEX encoding = UTF-8 Unicode
%%
%% Fontspec and font loading and configuration.
%%

%% XeLaTeX font support.
\usepackage{fontspec}
\usepackage{xltxtra}

%% To interrogate a font for its features use ‘otfinfo -f’.

%% The basic text font is the roman font.
%\setromanfont{Brill}[Path=./fonts/,
%  Extension      = .ttf,
%  UprightFont    = *-Roman,
%  BoldFont       = *-Bold,
%  ItalicFont     = *-Italic,
%  BoldItalicFont = *-BoldItalic]
%\setromanfont{Cambria.ttc}[Path=./fonts/,
%  UprightFeatures    = {FontIndex=0},
%  BoldFont           = Cambriab.ttf,
%  ItalicFont         = Cambriai.ttf,
%  BoldItalicFont     = Cambriaz.ttf,
%  Ligatures = Common,
%  Numbers = OldStyle]
\setromanfont{LibertinusSerif}[Path=./fonts/,
  Extension      = .otf,
  UprightFont    = *-Regular,
  ItalicFont     = *-Italic,
  BoldFont       = *-Bold,
  BoldItalicFont = *-BoldItalic,
  Ligatures      = Common,
  Numbers        = OldStyle]

%% Settings for all other fonts.
%% - MatchLowercase resizes to match the x-height of the roman font.
\defaultfontfeatures{Scale=MatchLowercase}

%% The sans-serif counterpart to the roman font.
%\setsansfont{Calibri}[Path=./fonts/,
%  Extension      = .ttf,
%  UprightFont    = *,
%  BoldFont       = *b,
%  ItalicFont     = *i,
%  BoldItalicFont = *z,
%  Ligatures      = Common,
%  LetterSpace    = -0.1]
%\setsansfont{LibertinusSans}[Path=./fonts/,
%  Extension      = .otf,
%  UprightFont    = *-Regular,
%  ItalicFont     = *-Italic,
%  BoldFont       = *-Bold,
%  BoldItalicFont = *-Italic,
%  Ligatures      = Common,
%  Scale          = 1]
\setsansfont{SourceSans3}[Path=./fonts/,
  Extension      = .otf,
  UprightFont    = *-Regular,
  ItalicFont     = *-It,
  BoldFont       = *-Semibold,
  BoldItalicFont = *-SemiboldIt,
  Ligatures      = Common,
  Numbers        = Lining,
  Scale          = 0.95]
 
%% The monospace counterpart to the roman font.
\setmonofont{Consolas}[Path=./fonts/,
  Extension      = .ttf,
  UprightFont    = consola,
  ItalicFont     = consolai,
  BoldFont       = consolab,
  BoldItalicFont = consolaz,
  FakeStretch    = 0.9]

%%
%% Special transformations derived from the base font.
%%

%% Condensed version of roman font.
%\newfontfamily{\romancondensedfont}{Brill}[Path=./fonts/,
%  Extension      = .ttf,
%  UprightFont    = *-Roman,
%  BoldFont       = *-Bold,
%  ItalicFont     = *-Italic,
%  BoldItalicFont = *-BoldItalic,
%  FakeStretch    = 0.94]
\newfontfamily{\romancondensedfont}{LibertinusSerif}[Path=./fonts/,
  Extension      = .otf,
  UprightFont    = *-Regular,
  ItalicFont     = *-Italic,
  BoldFont       = *-Bold,
  BoldItalicFont = *-BoldItalic,
  Ligatures       = Common,
  Numbers         = OldStyle,
  FakeStretch     = 0.94]

%% Gloss abbreviations.
%\newfontfamily{\glossfont}{Brill}[Path=./fonts/,
%  Extension      = .ttf,
%  UprightFont    = *-Roman,
%  BoldFont       = *-Bold,
%  ItalicFont     = *-Italic,
%  BoldItalicFont = *-BoldItalic,
%  FakeStretch    = 0.9]
\newfontfamily{\glossfont}{LibertinusSerif}[Path=./fonts/,
  Extension      = .otf,
  UprightFont    = *-Regular,
  ItalicFont     = *-Italic,
  BoldFont       = *-Bold,
  BoldItalicFont = *-BoldItalic,
  Ligatures       = Common,
  Numbers         = OldStyle,
  FakeStretch     = 0.9]

%%
%% Special fonts.
%%

%% Some of these font definitions are used by stuff in the
%% \FILEPREFIX-fonts-symbols.tex file that gets loaded after
%% this file.

%% Phonetics.
%\newfontfamily{\ipafont}{Charis SIL Compact}
\newfontfamily{\ipafont}{CharisSIL}[Path=./fonts/,
  Extension      = .ttf,
  UprightFont    = *-Regular,
  ItalicFont     = *-Italic,
  BoldFont       = *-Bold,
  BoldItalicFont = *-BoldItalic,
  Ligatures      = Common]


%% This font is for unusual phonetic symbols that might not
%% be available in the ipafont.
%\newfontfamily{\phonfont}{Charis SIL Compact}
\newfontfamily{\phonfont}{CharisSIL}[Path=./fonts/,
  Extension      = .ttf,
  UprightFont    = *-Regular,
  ItalicFont     = *-Italic,
  BoldFont       = *-Bold,
  BoldItalicFont = *-BoldItalic,
  Ligatures      = Common]

%% This font is for mathematical symbols.
%%
%% Don't call it \mathfont because other packages use that name.
%% You have been warned.
%\newfontfamily{\mthfont}{Asana Math}
\newfontfamily{\mthfont}{Asana-Math.ttf}[Path=./fonts/]

%% This font is for calligraphic script math symbols.
%\newfontfamily{\mthcalfont}{XITS Math}
\newfontfamily{\mthcalfont}{XITSMath}[Path=./fonts/,
  Extension      = .otf,
  UprightFont    = *-Regular,
  ItalicFont     = *-Regular,
  BoldFont       = *-Bold,
  BoldItalicFont = *-Bold,
  Ligatures      = Common]

%% This font is for the variant slashed zero form of U+2205 ∅ Empty Set.
%% 
%% There is only a regular and bold version of this font so we fake italics.
%\newfontfamily{\nullzerofont}{STIXVar}[Path=./fonts/,
%  Extension          = .otf,
%  UprightFont        = STIXVar,
%  Scale              = 1,
%  AutoFakeSlant      = 0.13,
%  BoldFont           = STIXVarBol,
%  BoldItalicFont     = STIXVarBol,
%  BoldItalicFeatures = {FakeSlant=0.13}]
\newfontfamily{\nullzerofont}{LibertinusSerif}[Path=./fonts/,
  Extension      = .otf,
  UprightFont    = *-Regular,
  ItalicFont     = *-Italic,
  BoldFont       = *-Bold,
  BoldItalicFont = *-BoldItalic,
  Ligatures       = Common,
  Numbers         = SlashedZero]

%% This font is for the check mark and ex mark for good and bad things.
%%
%% There is only a regular version of this font and we don’t want any
%% other forms so we use the same -Regular for all shapes.
\newfontfamily{\checkexfont}{NotoSansSymbols2}[Path=./fonts/,
  Extension      = .otf,
  UprightFont    = *-Regular,
  ItalicFont     = *-Regular,
  BoldFont       = *-Regular,
  BoldItalicFont = *-Regular,
  Ligatures       = Common]
